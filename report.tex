\documentclass[12pt]{article}

% Русский язык
\usepackage[english,russian]{babel}

% Красная строка
\usepackage{indentfirst}\frenchspacing

% Размер страницы и поля
\usepackage[a4paper, portrait, top=2cm, bottom=2cm, left=3cm, textwidth=16cm]{geometry}

% Шрифты
\usepackage{fontspec}
\setmainfont{Times New Roman}
% \newfontfamily\cyrillicfont[Mapping=tex-text,Script=Cyrillic]{Times New Roman}
% \setmonofont{Consolas}

% Интерлиньяж
\usepackage{setspace} 
\setstretch{1.2}

% Метаданные
\title{Отчет по учебной (технологической) практике}
\author{Сидор Петрович Иванов}
\date{\today}

% Гиперссылки
\usepackage{hyperref}
\hypersetup{colorlinks=true,linkcolor=black}

% Таблицы с автоподбором ширины столбцов
\usepackage{array}
\usepackage{tabularx}

\begin{document}

\thispagestyle{empty}
\begin{center}
  \small{
    Министерство науки и высшего образования Российской Федерации

    ФГАОУ ВО <<Северо-Востоный федеральный университет имени М.К. Аммосова>>

    Институт математики и информатики

    Кафедра <<Информационные технологии>>
  }

  \vfill
  
  \Large{
    \textbf{ОТЧЕТ}

    \textbf{Учебной (технологической) практики}
  }
\end{center}

\vfill

\begin{flushleft}
ФИО студента: \underline{Сидор Петрович Иванов}

Направление подготовки: \underline{09.03.01 Информатика и вычислительные технологии} 

Направленность: \underline{Технологии разработки программного обеспечения}

Курс обучения: \underline{2 (второй), группа Б-ИВТ-22-1}

Вид практики: \underline{Учебная (технологическая)}

Сроки прохождения практики: \underline{22 июня -- 7 июля 2024 г.}

Место прохождения практики: \underline{кафедра <<Информационные технологии>>}

Руководитель практики: \underline{Леверьев В. С., ст. преп. каф.  <<Информационные технологии>>} 
\end{flushleft}

\vfill

\begin{center}
  \small{Якутск, 2024}
\end{center}

\newpage \section*{Введение}

Введение должно быть кратким и содержать основную информацию: ФИО студента, курс, направление подготовки и вуз, место прохождения практики.

Здесь же указывают цели и задачи практики. Цели — то, чего хочется достичь: развить навыки, разобраться в рабочих процессах. Задачи — конкретные действия, которые вы должны выполнить по ходу учебной практики.

Также во введении можно добавить, почему стажировка важна для вашей учебы и будущей карьеры.

\section{Онлайн курс <<Latex с нуля>>}

\section{Онлайн-курс <<Основы работы с базами данных и SQL>>}

\section{Онлайн-курс <<Введение в Git>>}
	
\section{Составление реферата}

\section{Обработка текста для датасета}

\section*{Заключение}

\newpage \section*{Дневник учебно-эксплуатационной практики}

\begin{center}
  Календарный план
  \vspace{3ex}
\end{center}

\noindent
\begin{tabularx}{\textwidth}
{| >{\centering\arraybackslash}p{0.5cm} | >{\centering\arraybackslash}X | 
   >{\centering\arraybackslash}p{3.5cm} | >{\centering\arraybackslash}p{2.5cm} |}
\hline
\textbf{№} & \textbf{Содержание работ} & \textbf{Вид отчета} & \textbf{Даты выполнения работ} \\ \hline
1  & Прохождение курса <<LaTeX с нуля>> & Сертификат & 22.06 -- 24.06 \\ \hline
2  & Прохождение курса <<Основы работы с базами данных и SQL>> & Сертификат & 25.06 -- 27.06 \\ \hline
3  & Прохождение курса <<Введение в Git>> & Сертификат & 28.06 -- 30.06 \\ \hline
4  & Составление реферата & Проект в Overleaf & 01.07 -- 03.07  \\ \hline
5  & Обработка текста для датасета & Текстовые файлы & 04.07 -- 06.07 \\ \hline
6  & Оформление отчета & tex + pdf & 07.07.2024 \\ \hline
\end{tabularx}

\vspace{1cm}

\noindent
\begin{tabularx}{\textwidth}{ X >{\centering\arraybackslash}X >{\raggedleft\arraybackslash}X }
Подпись практиканта: & & / Иванов С.П. / \\
& $\overline{\parbox[t]{3cm}{\centering\footnotesize(подпись)}}$ \vspace{1cm} & \\
\multicolumn{3}{l}{Содержание и объём выполненных работ подтверждаю:} \\
Руководитель практики: & & / Леверьев В.С. / \\
& $\overline{\parbox[t]{3cm}{\centering\footnotesize(подпись)}}$ \vspace{1cm} & \\
Оценка практики: & \underline{\hspace{3cm}} & \\
& $\overline{\parbox[t]{3cm}{}}$ & \\
\end{tabularx}
    
\newpage \section*{Приложения}
\end{document}
