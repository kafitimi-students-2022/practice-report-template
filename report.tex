\documentclass[12pt]{article}

% Русский язык
\usepackage[english, russian]{babel}

% Красная строка
\usepackage{indentfirst}\frenchspacing

% Размер страницы и поля
\usepackage[a4paper, portrait, top=2cm, left=3cm, bottom=2cm, textwidth=16cm]{geometry}

% Шрифты
\usepackage{fontspec}
\setmainfont{Times New Roman}

% Интерлиньяж
\usepackage{setspace} 
\setstretch{1.2}

% Метаданные PDF и настройки гиперссылок
\usepackage[
  pdfauthor={<ФАМИЛИЯ ИМЯ ОТЧЕСТВО>},
  pdftitle={Отчет по учебной (технологической) практике}
]{hyperref}
\hypersetup{colorlinks=true, linkcolor=black}

% Таблицы с автоподбором ширины столбцов
\usepackage{array}
\usepackage{tabularx}

\begin{document}

\thispagestyle{empty}
\begin{center}
  Министерство науки и высшего образования Российской Федерации \\
  ФГАОУ ВО <<Северо-Востоный федеральный университет имени М.К. Аммосова>> \\
  Институт математики и информатики \\
  Кафедра <<Информационные технологии>>
\end{center}

  \vfill
  
\begin{center}
  \Large{
    \textbf{ОТЧЕТ} \\
    \textbf{учебной (технологической) практики}
  }
\end{center}

\vfill

\begin{flushleft}
Выполнил: \underline{<ФАМИЛИЯ ИМЯ ОТЧЕСТВО>} \\
Направление подготовки: \underline{<КОД И НАПРАВЛЕНИЕ>} \\
Направленность: \underline{<НАПРАВЛЕННОСТЬ>} \\
Группа: \underline{<ГРУППА>} \\
Вид практики: \underline{Учебная (технологическая)} \\
Сроки прохождения практики: \underline{22 июня -- 7 июля 2024 г.} \\
Место прохождения практики: \underline{кафедра <<Информационные технологии>>} \\
Руководитель практики: \underline{<ФАМИЛИЯ И.О.>, ст. преп. каф.  <<Информационные технологии>>}
\end{flushleft}

\vfill

\begin{center}
  Якутск, 2024
\end{center}

\newpage \section*{Введение}

Введение должно содержать основную вводную информацию: ФИО студента, группу, направление подготовки и вуз, вид и место прохождения практики.

Также необходимо указать цели и задачи практики. Цели --- это то, чего хочется достичь: развить навыки, разобраться в рабочих процессах. Задачи --- это конкретные действия, которые вы должны выполнить по ходу учебной практики.

Также во введении можно добавить, почему данная практика важна для вашей учебы и будущей карьеры.

Коды и названия направлений:

\noindent
\begin{tabularx}{\textwidth}{ | p{2.3cm} | p{1.5cm} | X | X | } \hline
Группа & Код & Направление & Направленность \\ \hline
Б-ФИИТ-22 & 02.03.02 & Фундаментальная информатика и информационные технологии & Программирование и информационные технологии \\ \hline
Б-ИВТ-22-1 & 09.03.01 & Информатика и вычислительная техника & Технологии разработки программного обеспечения \\ \hline
Б-ИВТ-22-2 & 09.03.01 & Информатика и вычислительная техника & Технологии разработки программного обеспечения \\ \hline
\end{tabularx}

\section{Онлайн-курс <<Latex с нуля>>}

Краткая характеристика онлайн-курса и описание приобретенных знаний, умений навыков.

\section{Онлайн-курс <<Основы работы с базами данных и SQL>>}

Краткая характеристика онлайн-курса и описание приобретенных знаний, умений навыков.

\section{Онлайн-курс <<Введение в Git>>}

Краткая характеристика онлайн-курса и описание приобретенных знаний, умений навыков.
	
\section{Составление реферата}

Тема реферата и что нового вы узнали при составлении реферата.

\section{Обработка текста для датасета}

Цель работы и объем выполненной работы.

\section*{Заключение}

В заключении, следует кратко описать выполненные задачи и достигнутые цели.

\newpage \section*{Дневник учебно-эксплуатационной практики}

\noindent
\begin{tabularx}{\textwidth}
{| >{\centering\arraybackslash}p{0.5cm} | >{\centering\arraybackslash}X | 
   >{\centering\arraybackslash}p{3.5cm} | >{\centering\arraybackslash}p{2.5cm} |}
\hline
\textbf{№} & \textbf{Содержание работ} & \textbf{Вид отчета} & \textbf{Даты выполнения работ} \\ \hline
1  & Прохождение курса <<LaTeX с нуля>> & Сертификат & 22.06 -- 24.06 \\ \hline
2  & Прохождение курса <<Основы работы с базами данных и SQL>> & Сертификат & 25.06 -- 27.06 \\ \hline
3  & Прохождение курса <<Введение в Git>> & Сертификат & 28.06 -- 30.06 \\ \hline
4  & Составление реферата & Проект в Overleaf & 01.07 -- 03.07  \\ \hline
5  & Обработка текста для датасета & Текстовые файлы & 04.07 -- 06.07 \\ \hline
6  & Оформление отчета & tex + pdf & 07.07.2024 \\ \hline
\end{tabularx}

\vspace{1cm}

\noindent
\begin{tabularx}{\textwidth}{ X >{\centering\arraybackslash}X >{\raggedleft\arraybackslash}X }
Подпись практиканта: & & / <ФАМИЛИЯ И.О.> / \\
& $\overline{\parbox[t]{3cm}{\centering\footnotesize(подпись)}}$ \vspace{1cm} & \\
\multicolumn{3}{l}{Содержание и объём выполненных работ подтверждаю:} \\
Руководитель практики: & & / <ФАМИЛИЯ И.О.> / \\
& $\overline{\parbox[t]{3cm}{\centering\footnotesize(подпись)}}$ \vspace{1cm} & \\
Оценка практики: & \underline{\hspace{3cm}} & \\
& $\overline{\parbox[t]{3cm}{}}$ & \\
\end{tabularx}
    
\newpage \section*{Приложения}
\end{document}
